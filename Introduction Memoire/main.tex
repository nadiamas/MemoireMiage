\documentclass{article}
\usepackage[utf8]{inputenc}

\title{Bucki – Analyse décisionnelle des Systèmes complexes (B-ADSC) }

\author{nadia.maslouhi }

\begin{document}

\maketitle

\section{Introduction}


B-ADSc : "Bucki – Analyse décisionnelle des Systèmes complexes" est une méthode dédiée à la conception et à l’analyse des systèmes et des organisations. Elle se distingue par la prise en compte effective des opérateurs humains avec leurs autonomies, leurs politiques de production, leurs procédés de prise de décision, leur retour d’expérience...

L'approche décisionnelle des systèmes complexes consiste en la conception, l'analyse et l'optimisation d'une organisation de telle sorte que les savoir-faire et les procédés employés puissent s’exprimer avec le plus d'efficacité et en totale convergence de buts avec la finalité de l'ensemble.

Dans la conception du Système d’Information, les flux des données et des traitements sont soumis aux flux des décisions : en cela B-ADSc généralise les analyses fonctionnelles.

Ainsi, le Système d’information est intimement lié à l’Organisation des acteurs (hommes, machines) jusqu’à se confondre.

C’est quoi B-ADSC plus précisément ? qu’elle la hiérarchie de ce concept ?

C’est quoi la différence entre B-ADSC et l’analyse fonctionnelle ?

Est-ce que l’analyse décisionnelle apporte des améliorations au niveau de l’efficacité des organisations ?

Est-ce que B-ADSC permet l’évolution du champ d’action des entreprises ? comment ?



\end{document}
