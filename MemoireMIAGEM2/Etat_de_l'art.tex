
 
\subsection{Self-Healing}

Nous considérons: (i) que les WS peuvent subir des défaillances silencieuses (les erreurs logiques ne sont pas prises en compte); (ii) le moteur d'exécution, en charge de l'exécution du CWS, fonctionne loin des serveurs WS dans des serveurs fiables tels que les clusters, il n'échoue pas, son réseau de données est hautement sécurisé et n'est pas affecté par les fautes WS; (iii) nous supposons que l'information nécessaire pour choisir une stratégie de récupération est connue par le moteur d'exécution "à tout moment pour chaque WS. 



\subsubsection{Exécution tolérante aux pannes des CWS}

Le contrôle d'exécution des Web services composites peut être centralisé c’est à dire un coordinateur qui va jouer le rôle de la gestion de toute l’exécution,  ou distribué dans lequel le processus d’exécution de déroule avec la collaboration de plusieurs participants sans un coordinateur central. Comme il peut être attaché aux web services composants ou indépendant.
Certaines méthodes indépendantes de tolérance aux pannes dont apparus, telles que les propriétés transactionnelles et la réplication. Les propriétés transactionnelles décrivent implicitement le comportement des services web en cas d'échec, et sont utilisées pour garantir la propriété transactionnelle d’atomicité.
